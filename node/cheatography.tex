\documentclass[10pt,a4paper]{article}

% Packages
\usepackage{fancyhdr}           % For header and footer
\usepackage{multicol}           % Allows multicols in tables
\usepackage{tabularx}           % Intelligent column widths
\usepackage{tabulary}           % Used in header and footer
\usepackage{hhline}             % Border under tables
\usepackage{graphicx}           % For images
\usepackage{xcolor}             % For hex colours
\usepackage[utf8x]{inputenc}    % For unicode character support
\usepackage[T1]{fontenc}        % Without this we get weird character replacements
\usepackage{colortbl}           % For coloured tables
\usepackage{setspace}           % For line height
\usepackage{lastpage}           % Needed for total page number
\usepackage{seqsplit}           % Splits long words.
%\usepackage{opensans}          % Can't make this work so far. Shame. Would be lovely.
\usepackage[normalem]{ulem}     % For underlining links
% Most of the following are not required for the majority
% of cheat sheets but are needed for some symbol support.
\usepackage{amsmath}            % Symbols
\usepackage{MnSymbol}           % Symbols
\usepackage{wasysym}            % Symbols
\usepackage[italian]{babel}              % Languages
\usepackage[nodayofweek]{datetime} % DATE
\usepackage{hyperref}
\hypersetup{
   colorlinks=true,
   linkcolor=black,
   filecolor=magenta,
   urlcolor=blue
}

% Document Info
\author{Luca Pascucci (lucapascucci)}
\pdfinfo{
   /Title (Cheat_Sheet_Node.pdf)
   /Creator (Luca Pascucci)
   /Author (Luca Pascucci (lucapascucci))
   /Subject (Cheat_Sheet_Node)
}

% Lengths and widths
\addtolength{\textwidth}{6cm}
\addtolength{\textheight}{-1cm}
\addtolength{\hoffset}{-3cm}
\addtolength{\voffset}{-2cm}
\setlength{\tabcolsep}{0.2cm} % Space between columns
\setlength{\headsep}{-12pt} % Reduce space between header and content
\setlength{\headheight}{85pt} % If less, LaTeX automatically increases it
\renewcommand{\footrulewidth}{0pt} % Remove footer line
\renewcommand{\headrulewidth}{0pt} % Remove header line
\renewcommand{\seqinsert}{\ifmmode\allowbreak\else\-\fi} % Hyphens in seqsplit
% This two commands together give roughly
% the right line height in the tables
\renewcommand{\arraystretch}{1.3}
\onehalfspacing

% Commands
\newcommand{\SetRowColor}[1]{\noalign{\gdef\RowColorName{#1}}\rowcolor{\RowColorName}} % Shortcut for row colour
\newcommand{\mymulticolumn}[3]{\multicolumn{#1}{>{\columncolor{\RowColorName}}#2}{#3}} % For coloured multi-cols
\newcolumntype{x}[1]{>{\raggedright}p{#1}} % New column types for ragged-right paragraph columns
\newcommand{\tn}{\tabularnewline} % Required as custom column type in use

% Font and Colours
\definecolor{HeadBackground}{HTML}{333333}
\definecolor{FootBackground}{HTML}{666666}
\definecolor{TextColor}{HTML}{333333}
\definecolor{DarkBackground}{HTML}{336600}
\definecolor{LightBackground}{HTML}{E6FFCC}
\renewcommand{\familydefault}{\sfdefault}
\color{TextColor}

% Header and Footer
\pagestyle{fancy}
\fancyhead{} % Set header to blank
\fancyfoot{} % Set footer to blank
\fancyhead[L]{
\noindent
\begin{multicols}{3}
\begin{tabulary}{5cm}{C}
   \SetRowColor{DarkBackground}
   \vspace{-6pt}
   {\parbox{\dimexpr\textwidth-2\fboxsep\relax}
      {
         \includegraphics[height=55px]{./images/Nodejs.png}
      }
   }
\end{tabulary}
\columnbreak
\begin{tabulary}{12cm}{L}
   \vspace{-2pt}\large{\bf{\textcolor{DarkBackground}{\textrm{Cheat Sheet Node}}}} \\
   \normalsize{by \textcolor{DarkBackground}{Luca Pascucci (lucapascucci)} via \textcolor{DarkBackground}{\uline{cheatography.com/lucapascucci/}}}
\end{tabulary}
\end{multicols}}

\fancyfoot[L]{ \footnotesize
\noindent
\begin{multicols}{3}
\begin{tabulary}{5.8cm}{LL}
   \SetRowColor{FootBackground}
   \mymulticolumn{2}{p{5.377cm}}{\bf\textcolor{white}{Cheatographer}}  \\
   \vspace{-2pt}Luca Pascucci (lucapascucci) \\
   \href{https://www.cheatography.com/lucapascucci/}{cheatography.com/lucapascucci} \\
   \end{tabulary}
\vfill
\columnbreak
\begin{tabulary}{5.8cm}{L}
   \SetRowColor{FootBackground}
   \mymulticolumn{1}{p{5.377cm}}{\bf\textcolor{white}{Cheat Sheet}}  \\
   \vspace{-2pt}Published 28 agosto 2019.\\
   Updated \today .\\
   Page {\thepage} of \pageref{LastPage}.
\end{tabulary}
\vfill
\columnbreak
\begin{tabulary}{5.8cm}{L}
   \SetRowColor{FootBackground}
   \mymulticolumn{1}{p{5.377cm}}{\bf\textcolor{white}{Sponsor}} \\
   \SetRowColor{white}
   \vspace{-5pt}
   \includegraphics[width=120px,height=48px]{./images/EBWorld.png} \\
   EBWolrd s.r.l.\\
   \href{https://www.ebw.it/}{ebw.it}
\end{tabulary}
\end{multicols}}


\begin{document}
\raggedright
\raggedcolumns

% Set font size to small. Switch to any value
% from this page to resize cheat sheet text:
% www.emerson.emory.edu/services/latex/latex_169.html
\footnotesize % Small font.

\begin{multicols*}{2}

   \begin{tabularx}{8.5cm}{X}
      \SetRowColor{DarkBackground}
      \bf\textcolor{white}{Configurazione workspace} \tn

      \SetRowColor{LightBackground}
      Copiare il file \href{https://drive.google.com/open?id=1iDtjWpuzGI1B5ZMyhMM1B_ALMNrUV70H}{\textbf{.editorconfig}} nella root del progetto \tn

      \SetRowColor{white}
      \textbf{Plugins:}
      \href{https://marketplace.visualstudio.com/items?itemName=EditorConfig.EditorConfig}{VSCode}
      \href{https://atom.io/packages/editorconfig}{Atom}
      \href{https://github.com/editorconfig/editorconfig-notepad-plus-plus}{Nodepad++} \tn

      \SetRowColor{LightBackground}
      Copiare il file \href{https://drive.google.com/open?id=1VeYpXkt4g76K1OmziXWIntxI5FgZjUmF}{\textbf{.gitignore}} nella root del progetto \tn

      \hhline{>{\arrayrulecolor{DarkBackground}}-}
   \end{tabularx}
   \par\addvspace{1em}

   \begin{tabularx}{8.5cm}{X}
      \SetRowColor{DarkBackground}
      \bf\textcolor{white}{ESLint (Aiuta durante la stesura del codice)}  \tn

      \SetRowColor{LightBackground}
      \textbf{Installazione} \tn

      \SetRowColor{white}
      npm install \texttt{-{}-}save-dev eslint \\
      npm install \texttt{-{}-}save-dev eslint-plugin-jsdoc \tn

      \SetRowColor{LightBackground}
      Copiare \href{https://drive.google.com/open?id=1VeYpXkt4g76K1OmziXWIntxI5FgZjUmF}{\textbf{.eslintrc}} nella root del progetto \tn

      \SetRowColor{white}
      \textbf{Plugins}
      \href{https://marketplace.visualstudio.com/items?itemName=dbaeumer.vscode-eslint}{VSCode}
      \href{https://atom.io/packages/eslint}{Atom} \tn

      \hhline{>{\arrayrulecolor{DarkBackground}}-}
   \end{tabularx}
   \par\addvspace{1em}

    \begin{tabularx}{8.5cm}{X}
      \SetRowColor{DarkBackground}
      \bf\textcolor{white}{JsDoc (Setup)} \tn

      \SetRowColor{LightBackground}
      \textbf{Installazione} \tn

      \SetRowColor{white}
         npm install \texttt{-{}-}save-dev jsdoc \\
         npm install \texttt{-{}-}save-dev toast-jsdoc \tn

      \SetRowColor{LightBackground}
      Copiare le configurazioni \href{https://drive.google.com/open?id=1VeYpXkt4g76K1OmziXWIntxI5FgZjUmF}{\textbf{jsdoc\_conf.json}} nella root del progetto \tn

      \SetRowColor{white}
      \textbf{Script da aggiungere nel package.json del progetto} \tn

      \SetRowColor{white}
         "createJSDoc": "jsdoc -c jsdoc\_conf.json \texttt{-{}-}verbose",\\
         "openJSDoc": "start chrome documentation\textbackslash{}\textbackslash{}index.html", \\
         "documentation": "npm run createJSDoc \&\& npm run openJSDoc" \tn

      \SetRowColor{LightBackground}
      \textbf{Inserire nel .gitignore} (se non presente) \tn

      \SetRowColor{white}
      \# Cartella documentazione JsDoc \\ documentation/ \tn

      \SetRowColor{LightBackground}
      \textbf{Creare e visualizzare documentazione} \tn

      \SetRowColor{white}
      npm run documentation \tn

      \hhline{>{\arrayrulecolor{DarkBackground}}-}
   \end{tabularx}
   \par\addvspace{1em}

   \begin{tabularx}{8.5cm}{X}
      \SetRowColor{DarkBackground}
      \bf\textcolor{white}{JsDoc (Regole documentazione)}  \tn

      \SetRowColor{LightBackground}
      \textbf{Intestazione di ogni file .js (UpperCamelCase)} \tn

      \SetRowColor{white}
      /** \textbf{@module} NomeModulo */ \tn

      \SetRowColor{LightBackground}
      \textbf{Require di moduli interni (nostri)} \tn

      \SetRowColor{white}
      /** \textbf{@see module:}NomeModulo */ \\
      const api = require('./nome\_file'); \tn

      \SetRowColor{LightBackground}
      \textbf{Funzioni} \tn

      \SetRowColor{white}
         /**\\
         * Commento della funzione\\
         * \textbf{@param} \{Type\} \textit{param1} Commento parametro 1\\
         * \textbf{@param} \{boolean\} \textit{param2} Commento parametro 2\\
         * \textbf{@return} \{number\} Commento sul valore di ritorno\\
         * \textbf{@example} <caption> Esempio espicativo di utilizzo della funzione </caption> \\
         * prova(4,true) = \dots \\
         */ \\
         function prova(param1,param2): number \{ \\
         \dots \\
         \} \tn

      \hhline{>{\arrayrulecolor{DarkBackground}}-}
   \end{tabularx}
   \par\addvspace{1em}

   \begin{tabularx}{8.5cm}{X}
      \SetRowColor{DarkBackground}
      \bf\textcolor{white}{REST API}  \tn

      \SetRowColor{LightBackground}
      \textbf{Funzioni} \tn

      \SetRowColor{white}
      /**\\
      * Commento della funzione\\
      * \textbf{@param} \{*\} \textit{request} richiesta\\
      * \textbf{@param} \{*\} \textit{response} risposta\\
      * \textbf{@example} <caption> Body della richiesta </caption> \\
      * request.body = \{ "id": 1023 \} \\
      */ \\
      function provaAPI(request,response) \{ \\
      \begin{enumerate}
         \item Controllo dei parametri (in caso errori rispondere con una EBW\_API\_response)
         \item Esecuzione API
         \item Risposta (formattata come EBW\_API\_response)
      \end{enumerate}
      \} \tn

      \hhline{>{\arrayrulecolor{DarkBackground}}-}
   \end{tabularx}
   \par\addvspace{1em}

   \begin{tabularx}{8.5cm}{X}
      \SetRowColor{DarkBackground}
      \bf\textcolor{white}{EBW API Response}  \tn

      \SetRowColor{LightBackground}
      \textbf{Importazione} \tn

      \SetRowColor{white}
      /** \textbf{@see module:}EBW\_API\_Response\_Factory */ \\
      const EBWAPIResponseFactory = require('./EBW\_API\_response\_factory'); \tn

      \SetRowColor{LightBackground}
      \textbf{Utilizzo} \tn

      \SetRowColor{white}
      new EBWAPIResponseFactory.createResponse(\\
      \{\\
      error: 0, (dipende dal tipo di risposta)\\
      status: 0, (dipende dal tipo di risposta)\\
      message: 'messaggio esplicativo', \\
      data: \{ \dots \}, \\
      traceback: \{ \dots \} \\
      \}\\
      ); \tn

      \hhline{>{\arrayrulecolor{DarkBackground}}-}
   \end{tabularx}
   \par\addvspace{1em}

% \begin{tabularx}{5.377cm}{X}
% \SetRowColor{DarkBackground}
% \mymulticolumn{1}{x{5.377cm}}{\bf\textcolor{white}{\{\{fa-columns\}\} Columns}}  \tn
% % Row 0
% \SetRowColor{LightBackground}
% \mymulticolumn{1}{x{5.377cm}}{Blocks are organised into columns by you.} \tn
% % Row Count 1 (+ 1)
% % Row 1
% \SetRowColor{white}
% \mymulticolumn{1}{x{5.377cm}}{PDFs organise blocks into columns automatically.} \tn
% % Row Count 2 (+ 1)
% % Row 2
% \SetRowColor{LightBackground}
% \mymulticolumn{1}{x{5.377cm}}{Try to keep the columns roughly even in length - it makes the cheat sheets easier to use online.} \tn
% % Row Count 4 (+ 2)
% \hhline{>{\arrayrulecolor{DarkBackground}}-}
% \end{tabularx}
% \par\addvspace{1.3em}

% \begin{tabularx}{5.377cm}{X}
% \SetRowColor{DarkBackground}
% \mymulticolumn{1}{x{5.377cm}}{\bf\textcolor{white}{\{\{fa-file-text-o\}\} Plain Text Block}}  \tn
% \SetRowColor{white}
% \mymulticolumn{1}{x{5.377cm}}{This is an example of a plain text block, filled with example content. Text in these blocks is added as-is, including \newline % Row Count 3 (+ 3)
% line breaks.% Row Count 4 (+ 1)
% } \tn
% \hhline{>{\arrayrulecolor{DarkBackground}}-}
% \end{tabularx}
% \par\addvspace{1.3em}

% \begin{tabularx}{5.377cm}{X}
% \SetRowColor{DarkBackground}
% \mymulticolumn{1}{x{5.377cm}}{\bf\textcolor{white}{\{\{fa-code\}\} Code Block}}  \tn
% \SetRowColor{LightBackground}
% \mymulticolumn{1}{x{5.377cm}}{\textless{}?php \newline   // This is a code block. \newline   // It preserves indentation and  \newline   // uses a monospaced font.} \tn
% \hhline{>{\arrayrulecolor{DarkBackground}}-}
% \end{tabularx}
% \par\addvspace{1.3em}

% \begin{tabularx}{5.377cm}{X}
% \SetRowColor{DarkBackground}
% \mymulticolumn{1}{x{5.377cm}}{\bf\textcolor{white}{\{\{fa-columns\}\} One Column Block}}  \tn
% % Row 0
% \SetRowColor{LightBackground}
% \mymulticolumn{1}{x{5.377cm}}{This is a one-column block.} \tn
% % Row Count 1 (+ 1)
% % Row 1
% \SetRowColor{white}
% \mymulticolumn{1}{x{5.377cm}}{It's useful for listing long items.} \tn
% % Row Count 2 (+ 1)
% % Row 2
% \SetRowColor{LightBackground}
% \mymulticolumn{1}{x{5.377cm}}{It can also be useful for sequences of steps or commands in programming cheat sheets or recipes.} \tn
% % Row Count 4 (+ 2)
% % Row 3
% \SetRowColor{white}
% \mymulticolumn{1}{x{5.377cm}}{Rows have alternating background colours to make the cheat sheet easier to read.} \tn
% % Row Count 6 (+ 2)
% \hhline{>{\arrayrulecolor{DarkBackground}}-}
% \end{tabularx}
% \par\addvspace{1.3em}

% \begin{tabularx}{5.377cm}{x{1.74195 cm} x{3.23505 cm} }
% \SetRowColor{DarkBackground}
% \mymulticolumn{2}{x{5.377cm}}{\bf\textcolor{white}{\{\{fa-columns\}\} Two Column Block}}  \tn
% % Row 0
% \SetRowColor{LightBackground}
% Block Type: & Two Columns \tn
% % Row Count 1 (+ 1)
% % Row 1
% \SetRowColor{white}
% Useful For: & Translations \tn
% % Row Count 2 (+ 1)
% % Row 2
% \SetRowColor{LightBackground}
%  & Key / Value Pairs \tn
% % Row Count 3 (+ 1)
% % Row 3
% \SetRowColor{white}
%  & Lists of small items \tn
% % Row Count 4 (+ 1)
% % Row 4
% \SetRowColor{LightBackground}
% \mymulticolumn{2}{x{5.377cm}}{\{\{bt\}\}{\bf{Multi-Column Items}}} \tn
% % Row Count 5 (+ 1)
% % Row 5
% \SetRowColor{white}
% \mymulticolumn{2}{x{5.377cm}}{If you have an item in column 1 but not column 2, it will span both columns.} \tn
% % Row Count 7 (+ 2)
% \hhline{>{\arrayrulecolor{DarkBackground}}--}
% \end{tabularx}
% \par\addvspace{1.3em}

% \begin{tabularx}{5.377cm}{p{1.55618 cm} p{1.51041 cm} p{1.51041 cm} }
% \SetRowColor{DarkBackground}
% \mymulticolumn{3}{x{5.377cm}}{\bf\textcolor{white}{\{\{fa-columns\}\} Three Column Block}}  \tn
% % Row 0
% \SetRowColor{LightBackground}
% 1 & 1 & 2 \tn
% % Row Count 1 (+ 1)
% % Row 1
% \SetRowColor{white}
% 3 & 5 & 8 \tn
% % Row Count 2 (+ 1)
% % Row 2
% \SetRowColor{LightBackground}
% 13 & 21 & 34 \tn
% % Row Count 3 (+ 1)
% % Row 3
% \SetRowColor{white}
% \mymulticolumn{3}{x{5.377cm}}{55} \tn
% % Row Count 4 (+ 1)
% \hhline{>{\arrayrulecolor{DarkBackground}}---}
% \SetRowColor{LightBackground}
% \mymulticolumn{3}{x{5.377cm}}{These blocks are great for saving space when using lists of small items. \newline  \newline Also, this is the first example of a "content note", which appears under a content block.}  \tn
% \hhline{>{\arrayrulecolor{DarkBackground}}---}
% \end{tabularx}
% \par\addvspace{1.3em}

% \begin{tabularx}{5.377cm}{p{1.04425 cm} p{1.04425 cm} p{1.04425 cm} p{1.04425 cm} }
% \SetRowColor{DarkBackground}
% \mymulticolumn{4}{x{5.377cm}}{\bf\textcolor{white}{\{\{fa-columns\}\} Four Column Block}}  \tn
% % Row 0
% \SetRowColor{LightBackground}
% A & B & C & D \tn
% % Row Count 1 (+ 1)
% % Row 1
% \SetRowColor{white}
% E & F & G & H \tn
% % Row Count 2 (+ 1)
% % Row 2
% \SetRowColor{LightBackground}
% I & J & K & L \tn
% % Row Count 3 (+ 1)
% % Row 3
% \SetRowColor{white}
% M & N & O & P \tn
% % Row Count 4 (+ 1)
% % Row 4
% \SetRowColor{LightBackground}
% Q & R & S & T \tn
% % Row Count 5 (+ 1)
% % Row 5
% \SetRowColor{white}
% U & V & W & X \tn
% % Row Count 6 (+ 1)
% % Row 6
% \SetRowColor{LightBackground}
% Y & Z &  &  \tn
% % Row Count 7 (+ 1)
% \hhline{>{\arrayrulecolor{DarkBackground}}----}
% \SetRowColor{LightBackground}
% \mymulticolumn{4}{x{5.377cm}}{Much like the three column block, the four column block is handy for saving space when using lists of small items.}  \tn
% \hhline{>{\arrayrulecolor{DarkBackground}}----}
% \end{tabularx}
% \par\addvspace{1.3em}

% \begin{tabularx}{5.377cm}{X}
% \SetRowColor{DarkBackground}
% \mymulticolumn{1}{x{5.377cm}}{\bf\textcolor{white}{\{\{fa-question\}\} Question and Answer Block}}  \tn
% % Row 0
% \SetRowColor{LightBackground}
% \mymulticolumn{1}{x{5.377cm}}{What is this?} \tn
% \mymulticolumn{1}{x{5.377cm}}{\hspace*{6 px}\rule{2px}{6px}\hspace*{6 px}A Q\&A block displays each question with the answer indented underneath.} \tn
% % Row Count 3 (+ 3)
% % Row 1
% \SetRowColor{white}
% \mymulticolumn{1}{x{5.377cm}}{When is it useful?} \tn
% \mymulticolumn{1}{x{5.377cm}}{\hspace*{6 px}\rule{2px}{6px}\hspace*{6 px}It's great for question and answer content!} \tn
% % Row Count 5 (+ 2)
% \hhline{>{\arrayrulecolor{DarkBackground}}-}
% \end{tabularx}
% \par\addvspace{1.3em}

% \begin{tabularx}{5.377cm}{x{2.05965 cm} x{2.01388 cm} p{0.50347 cm} }
% \SetRowColor{DarkBackground}
% \mymulticolumn{3}{x{5.377cm}}{\bf\textcolor{white}{\{\{fa-bar-chart\}\} Bar Chart Block}}  \tn
% % Row 0
% \SetRowColor{white}
% Red & \hspace*{0.02 cm}\rule{0.3cm}{6px} & 12 \tn
% % Row 1
% \SetRowColor{LightBackground}
% Green & \hspace*{0.02 cm}\rule{0.56cm}{6px} & 24 \tn
% % Row 2
% \SetRowColor{white}
% Orange & \hspace*{0.02 cm}\rule{2.02cm}{6px} & 88 \tn
% % Row 3
% \SetRowColor{LightBackground}
% Blue & \hspace*{0.02 cm}\rule{1.16cm}{6px} & 50 \tn
% \hhline{>{\arrayrulecolor{DarkBackground}}---}
% \SetRowColor{LightBackground}
% \mymulticolumn{3}{x{5.377cm}}{The bar chart block takes labels and numbers and generates a chart like the above.}  \tn
% \hhline{>{\arrayrulecolor{DarkBackground}}---}
% \end{tabularx}
% \par\addvspace{1.3em}

% \begin{tabularx}{5.377cm}{X}
% \SetRowColor{DarkBackground}
% \mymulticolumn{1}{x{5.377cm}}{\bf\textcolor{white}{\{\{fa-gift\}\} Tips}}  \tn
% % Row 0
% \SetRowColor{LightBackground}
% \mymulticolumn{1}{x{5.377cm}}{Use {\bf{tags}} - they help people find your cheat sheet!} \tn
% % Row Count 2 (+ 2)
% % Row 1
% \SetRowColor{white}
% \mymulticolumn{1}{x{5.377cm}}{Pick a {\bf{dark colour}} - it makes your cheat sheet easier to read.} \tn
% % Row Count 4 (+ 2)
% % Row 2
% \SetRowColor{LightBackground}
% \mymulticolumn{1}{x{5.377cm}}{Write a good {\bf{description}} so people know what's on your cheat sheet.} \tn
% % Row Count 6 (+ 2)
% % Row 3
% \SetRowColor{white}
% \mymulticolumn{1}{x{5.377cm}}{{\bf{Tell people}} about it when you're finished!} \tn
% % Row Count 7 (+ 1)
% \hhline{>{\arrayrulecolor{DarkBackground}}-}
% \end{tabularx}
% \par\addvspace{1.3em}

% \begin{tabularx}{5.377cm}{X}
% \SetRowColor{DarkBackground}
% \mymulticolumn{1}{x{5.377cm}}{\bf\textcolor{white}{\{\{fa-file-image-o\}\} Image Block}}  \tn
% \SetRowColor{LightBackground}
% \mymulticolumn{1}{p{5.377cm}}{\vspace{1px}\centerline{\includegraphics[width=5.1cm]{./images/EBWorld.png}}} \tn
% \hhline{>{\arrayrulecolor{DarkBackground}}-}
% \SetRowColor{LightBackground}
% \mymulticolumn{1}{x{5.377cm}}{The image block is just an image, displayed in a block. In the live version of a cheat sheet (not the PDF) you can click on an image for full size.}  \tn
% \hhline{>{\arrayrulecolor{DarkBackground}}-}
% \end{tabularx}
% \par\addvspace{1.3em}

% \begin{tabularx}{5.377cm}{X}
% \SetRowColor{DarkBackground}
% \mymulticolumn{1}{x{5.377cm}}{\bf\textcolor{white}{\{\{fa-film\}\} Video Block}}  \tn
% \SetRowColor{LightBackground}
% \mymulticolumn{1}{p{5.377cm}}{ Video: http://youtu.be/3ZqPaohVjmw} \tn
% \hhline{>{\arrayrulecolor{DarkBackground}}-}
% \SetRowColor{LightBackground}
% \mymulticolumn{1}{x{5.377cm}}{Video can be pulled in from YouTube or Vimeo, and will be shown on the website but not on the PDFs.}  \tn
% \hhline{>{\arrayrulecolor{DarkBackground}}-}
% \end{tabularx}
% \par\addvspace{1.3em}

% \begin{tabularx}{5.377cm}{x{2.4885 cm} x{2.4885 cm} }
% \SetRowColor{DarkBackground}
% \mymulticolumn{2}{x{5.377cm}}{\bf\textcolor{white}{\{\{fa-calendar\}\} Live Content Block}}  \tn
% % Row 0
% \SetRowColor{LightBackground}
% {\bf{Label}} & {\bf{Value}} \tn
% % Row Count 1 (+ 1)
% % Row 1
% \SetRowColor{white}
% Year & 2018 \tn
% % Row Count 2 (+ 1)
% % Row 2
% \SetRowColor{LightBackground}
% Month & 9 \tn
% % Row Count 3 (+ 1)
% % Row 3
% \SetRowColor{white}
% Day & 26 \tn
% % Row Count 4 (+ 1)
% \hhline{>{\arrayrulecolor{DarkBackground}}--}
% \SetRowColor{LightBackground}
% \mymulticolumn{2}{x{5.377cm}}{A "live content" block pulls its content from a URL, in a specific format. It's great for anything that changes regularly, like live statistics, up-to-the-minute sports results and tables, stock and share prices, TV schedules and even traffic and weather reports.}  \tn
% \hhline{>{\arrayrulecolor{DarkBackground}}--}
% \end{tabularx}
% \par\addvspace{1.3em}

% \begin{tabularx}{5.377cm}{x{1.33664 cm} x{1.08602 cm} x{1.16956 cm} p{0.58478 cm} }
% \SetRowColor{DarkBackground}
% \mymulticolumn{4}{x{5.377cm}}{\bf\textcolor{white}{\{\{fa-check\}\} Icons}}  \tn
% % Row 0
% \SetRowColor{LightBackground}
% \{\{literal\}\}\{\{fa-rocket\}\} & \{\{br\}\}\{\{fa-rocket\}\} & \{\{literal\}\}\{\{fa-bold\}\} & \{\{fa-bold\}\} \tn
% % Row Count 3 (+ 3)
% % Row 1
% \SetRowColor{white}
% \{\{literal\}\}\{\{fa-check\}\} & \{\{br\}\}\{\{fa-check\}\} & \{\{literal\}\}\{\{fa-leaf\}\} & \{\{fa-leaf\}\} \tn
% % Row Count 6 (+ 3)
% % Row 2
% \SetRowColor{LightBackground}
% \{\{literal\}\}\{\{fa-square-o\}\} & \{\{br\}\}\{\{fa-square-o\}\} & \{\{literal\}\}\{\{fa-cut\}\} & \{\{fa-cut\}\} \tn
% % Row Count 9 (+ 3)
% % Row 3
% \SetRowColor{white}
% \mymulticolumn{4}{x{5.377cm}}{\{\{bt\}\}Check out the \{\{popup="http://fortawesome.github.io/Font-Awesome/icons/"\}\}full list of Font Awesome icons\{\{/popup\}\}!} \tn
% % Row Count 12 (+ 3)
% \hhline{>{\arrayrulecolor{DarkBackground}}----}
% \SetRowColor{LightBackground}
% \mymulticolumn{4}{x{5.377cm}}{Icons work in block titles as well as content. There is a cheat sheet for Font Awesome icons at \seqsplit{http://www.cheatography.com/davechild/cheat-sheets/font-awesome/}}  \tn
% \hhline{>{\arrayrulecolor{DarkBackground}}----}
% \end{tabularx}
% \par\addvspace{1.3em}

% \begin{tabularx}{5.377cm}{x{1.9908 cm} x{2.9862 cm} }
% \SetRowColor{DarkBackground}
% \mymulticolumn{2}{x{5.377cm}}{\bf\textcolor{white}{\{\{fa-align-left\}\} Borders and Alignment}}  \tn
% % Row 0
% \SetRowColor{LightBackground}
% \{\{literal\}\}\{\{ac\}\} & \{\{ac\}\}Align text centrally \tn
% % Row Count 2 (+ 2)
% % Row 1
% \SetRowColor{white}
% \{\{literal\}\}\{\{ar\}\} & \{\{ar\}\}Align text right \tn
% % Row Count 4 (+ 2)
% % Row 2
% \SetRowColor{LightBackground}
% \{\{literal\}\}\{\{bb\}\} & \{\{bb\}\}Add border at bottom \tn
% % Row Count 6 (+ 2)
% % Row 3
% \SetRowColor{white}
% \{\{literal\}\}\{\{br\}\} & \{\{br\}\}Add border on right \tn
% % Row Count 8 (+ 2)
% % Row 4
% \SetRowColor{LightBackground}
% \{\{literal\}\}\{\{bl\}\} & \{\{bl\}\}Add border on left \tn
% % Row Count 10 (+ 2)
% % Row 5
% \SetRowColor{white}
% \{\{literal\}\}\{\{bt\}\} & \{\{bt\}\}Add border at top \tn
% % Row Count 12 (+ 2)
% \hhline{>{\arrayrulecolor{DarkBackground}}--}
% \end{tabularx}
% \par\addvspace{1.3em}

% \begin{tabularx}{5.377cm}{x{2.93643 cm} x{2.04057 cm} }
% \SetRowColor{DarkBackground}
% \mymulticolumn{2}{x{5.377cm}}{\bf\textcolor{white}{\{\{fa-bold\}\} Formatting Options}}  \tn
% % Row 0
% \SetRowColor{LightBackground}
% Italics ({\emph{example}}) & \{\{literal\}\}{\emph{text}} \tn
% % Row Count 2 (+ 2)
% % Row 1
% \SetRowColor{white}
% Bold ({\bf{example}}) & \{\{literal\}\}{\bf{text}} \tn
% % Row Count 4 (+ 2)
% % Row 2
% \SetRowColor{LightBackground}
% Superscript (\textasciicircum{}example\textasciicircum{}) & \{\{literal\}\}\textasciicircum{}text\textasciicircum{} \tn
% % Row Count 6 (+ 2)
% % Row 3
% \SetRowColor{white}
% Strikethrough (\textasciitilde{}\textasciitilde{}example\textasciitilde{}\textasciitilde{}) & \{\{literal\}\}\textasciitilde{}\textasciitilde{}text\textasciitilde{}\textasciitilde{} \tn
% % Row Count 8 (+ 2)
% % Row 4
% \SetRowColor{LightBackground}
% Inline Code (`code`) & \{\{literal\}\}`code` \tn
% % Row Count 10 (+ 2)
% % Row 5
% \SetRowColor{white}
% Asterisk (*) & \{\{literal\}\}* \tn
% % Row Count 11 (+ 1)
% % Row 6
% \SetRowColor{LightBackground}
% Carat (\textasciicircum{}) & \{\{literal\}\}\textasciicircum{} \tn
% % Row Count 12 (+ 1)
% \hhline{>{\arrayrulecolor{DarkBackground}}--}
% \SetRowColor{LightBackground}
% \mymulticolumn{2}{x{5.377cm}}{Sometimes you want to use a "*" or "\textasciicircum{}" in your content. Add a backslash before those characters (as an escape character) to tell Cheatography to treat them as normal text instead of special formatting characters, like this: \textbackslash{}* \textbackslash{}\textasciicircum{}}  \tn
% \hhline{>{\arrayrulecolor{DarkBackground}}--}
% \end{tabularx}
% \par\addvspace{1.3em}

% \begin{tabularx}{5.377cm}{x{1.74195 cm} x{3.23505 cm} }
% \SetRowColor{DarkBackground}
% \mymulticolumn{2}{x{5.377cm}}{\bf\textcolor{white}{\{\{fa-align-left\}\} Breaks and Spacing}}  \tn
% % Row 0
% \SetRowColor{LightBackground}
% \{\{literal\}\}~ & Insert non-breaking space \tn
% % Row Count 2 (+ 2)
% % Row 1
% \SetRowColor{white}
% \{\{literal\}\}\{\{nl\}\} & Line break \tn
% % Row Count 4 (+ 2)
% % Row 2
% \SetRowColor{LightBackground}
% \{\{literal\}\}\{\{noshy\}\} & Do not insert shy hyphens \tn
% % Row Count 6 (+ 2)
% % Row 3
% \SetRowColor{white}
% \{\{literal\}\}\{\{nobreak\}\} & Do not break long lines (only in list blocks) \tn
% % Row Count 8 (+ 2)
% % Row 4
% \SetRowColor{LightBackground}
% \{\{literal\}\}\{\{literal\}\} & Do not format text \tn
% % Row Count 10 (+ 2)
% \hhline{>{\arrayrulecolor{DarkBackground}}--}
% \end{tabularx}
% \par\addvspace{1.3em}

% \begin{tabularx}{5.377cm}{x{3.08574 cm} x{1.89126 cm} }
% \SetRowColor{DarkBackground}
% \mymulticolumn{2}{x{5.377cm}}{\bf\textcolor{white}{\{\{fa-external-link\}\} Links}}  \tn
% % Row 0
% \SetRowColor{LightBackground}
% \{\{literal\}\}\{\{popup="http://..."\}\} & Popup URL provided \tn
% % Row Count 2 (+ 2)
% % Row 1
% \SetRowColor{white}
% \mymulticolumn{2}{x{5.377cm}}{\{\{literal\}\}\{\{popup="http://..."\}\}link text\{\{/popup\}\}} \tn
% % Row Count 4 (+ 2)
% % Row 2
% \SetRowColor{LightBackground}
% \{\{literal\}\}\{\{link="http://..."\}\} & Link to URL provided \tn
% % Row Count 6 (+ 2)
% % Row 3
% \SetRowColor{white}
% \mymulticolumn{2}{x{5.377cm}}{\{\{literal\}\}\{\{link="http://..."\}\}link text\{\{/link\}\}} \tn
% % Row Count 7 (+ 1)
% \hhline{>{\arrayrulecolor{DarkBackground}}--}
% \end{tabularx}
% \par\addvspace{1.3em}

% \begin{tabularx}{5.377cm}{X}
% \SetRowColor{DarkBackground}
% \mymulticolumn{1}{x{5.377cm}}{\bf\textcolor{white}{\{\{fa-bullhorn\}\} Publishing}}  \tn
% % Row 0
% \SetRowColor{LightBackground}
% \mymulticolumn{1}{x{5.377cm}}{You can only publish a cheat sheet once.} \tn
% % Row Count 1 (+ 1)
% % Row 1
% \SetRowColor{white}
% \mymulticolumn{1}{x{5.377cm}}{Before that, only you can see it, unless you share the "preview" URL.} \tn
% % Row Count 3 (+ 2)
% % Row 2
% \SetRowColor{LightBackground}
% \mymulticolumn{1}{x{5.377cm}}{Newly published cheat sheets are prominent on Cheatography, so make the most of publishing and wait until you're happy with the cheat sheet!} \tn
% % Row Count 6 (+ 3)
% % Row 3
% \SetRowColor{white}
% \mymulticolumn{1}{x{5.377cm}}{You can still edit a cheat sheet after you've published it.} \tn
% % Row Count 8 (+ 2)
% % Row 4
% \SetRowColor{LightBackground}
% \mymulticolumn{1}{x{5.377cm}}{You can't unpublish a cheat sheet once it's been published.} \tn
% % Row Count 10 (+ 2)
% \hhline{>{\arrayrulecolor{DarkBackground}}-}
% \end{tabularx}
% \par\addvspace{1.3em}

% \begin{tabularx}{5.377cm}{X}
% \SetRowColor{DarkBackground}
% \mymulticolumn{1}{x{5.377cm}}{\bf\textcolor{white}{\{\{fa-language\}\} Language Support}}  \tn
% % Row 0
% \SetRowColor{LightBackground}
% \mymulticolumn{1}{x{5.377cm}}{Cheatography supports all languages, as best we can manage.} \tn
% % Row Count 2 (+ 2)
% % Row 1
% \SetRowColor{white}
% \mymulticolumn{1}{x{5.377cm}}{This means that the web versions of cheat sheets support all of the languages we've seen so far, including Chinese and Japanese.} \tn
% % Row Count 5 (+ 3)
% % Row 2
% \SetRowColor{LightBackground}
% \mymulticolumn{1}{x{5.377cm}}{We're working on right-to-left language support.} \tn
% % Row Count 6 (+ 1)
% \hhline{>{\arrayrulecolor{DarkBackground}}-}
% \end{tabularx}
% \par\addvspace{1.3em}

% \begin{tabularx}{5.377cm}{x{2.4885 cm} x{2.4885 cm} }
% \SetRowColor{DarkBackground}
% \mymulticolumn{2}{x{5.377cm}}{\bf\textcolor{white}{\{\{fa-star\}\} Rating Guidelines}}  \tn
% % Row 0
% \SetRowColor{LightBackground}
% \mymulticolumn{2}{x{5.377cm}}{Not sure what rating to give? Here are some tips to help you rate fairly and consistently.} \tn
% % Row Count 2 (+ 2)
% % Row 1
% \SetRowColor{white}
% \{\{nobreak\}\}\{\{fa-star\}\}\{\{fa-star\}\}\{\{fa-star\}\}\{\{fa-star\}\}\{\{fa-star\}\} & Superb. A great guide, likely to be useful to anyone interested in the topic. Probably well presented, at least a page long, and uses tags. \tn
% % Row Count 9 (+ 7)
% % Row 2
% \SetRowColor{LightBackground}
% \{\{nobreak\}\}\{\{fa-star\}\}\{\{fa-star\}\}\{\{fa-star\}\}\{\{fa-star\}\}\{\{fa-star-o\}\} & A good cheat sheet, with lots of useful information. Worth printing out a copy! \tn
% % Row Count 13 (+ 4)
% % Row 3
% \SetRowColor{white}
% \{\{nobreak\}\}\{\{fa-star\}\}\{\{fa-star\}\}\{\{fa-star\}\}\{\{fa-star-o\}\}\{\{fa-star-o\}\} & A decent cheat sheet, but missing some useful information. A useful reference, but could do with some improvement. \tn
% % Row Count 19 (+ 6)
% % Row 4
% \SetRowColor{LightBackground}
% \{\{nobreak\}\}\{\{fa-star\}\}\{\{fa-star\}\}\{\{fa-star-o\}\}\{\{fa-star-o\}\}\{\{fa-star-o\}\} & Room for improvement - probably not useful yet, but has potential with some more work. \tn
% % Row Count 24 (+ 5)
% % Row 5
% \SetRowColor{white}
% \{\{nobreak\}\}\{\{fa-star\}\}\{\{fa-star-o\}\}\{\{fa-star-o\}\}\{\{fa-star-o\}\}\{\{fa-star-o\}\} & A poor cheat sheet, likely accidentally published too soon, doesn't have much useful content, contains abuse, or with other serious problems. \tn
% % Row Count 32 (+ 8)
% \hhline{>{\arrayrulecolor{DarkBackground}}--}
% \end{tabularx}
% \par\addvspace{1.3em}

% \begin{tabularx}{5.377cm}{X}
% \SetRowColor{DarkBackground}
% \mymulticolumn{1}{x{5.377cm}}{\bf\textcolor{white}{\{\{fa-star\}\} Rating Tips}}  \tn
% % Row 0
% \SetRowColor{LightBackground}
% \mymulticolumn{1}{x{5.377cm}}{If someone has rated one of your cheat sheets, please consider rating one of theirs.} \tn
% % Row Count 2 (+ 2)
% % Row 1
% \SetRowColor{white}
% \mymulticolumn{1}{x{5.377cm}}{Avoid "revenge ratings". If you receive a low rating, bear in mind that not everyone will find your cheat sheets as useful as you do!} \tn
% % Row Count 5 (+ 3)
% % Row 2
% \SetRowColor{LightBackground}
% \mymulticolumn{1}{x{5.377cm}}{Feel free to leave a shout for another user to ask for more information about why they left the rating they did.} \tn
% % Row Count 8 (+ 3)
% \hhline{>{\arrayrulecolor{DarkBackground}}-}
% \end{tabularx}
% \par\addvspace{1.3em}


% That's all folks
\end{multicols*}

\end{document}
